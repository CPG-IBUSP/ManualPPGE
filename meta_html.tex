%%% Pacotes utilizados %%%

%% Codificação e formatação básica do LaTeX
% Suporte para português (hifenação e caracteres especiais)
\usepackage[english,brazilian]{babel}
% Aumenta fontes
\usepackage{scalefnt}

%Muda estilos de headings
\usepackage{sectsty}

% Codificação do arquivo
\usepackage[utf8]{inputenc}

% Mapear caracteres especiais no PDF
%\usepackage{cmap}

% Codificação da fonte
\usepackage[T1]{fontenc}

% Essencial para colocar funções e outros símbolos matemáticos
\usepackage{amsmath,amssymb,amsfonts,textcomp}

% Notas de rodapé com mesmo texto repetidas
%\usepackage{footmisc}

%% Layout
% Customização do layout da página, margens espelhadas
%\usepackage[twoside]{geometry}
% Aumenta as margens internas para espiral
%\geometry{bindingoffset=10pt}
%\geometry{textheight=610pt}
%\addtolength{\voffset}{25pt}
%\addtolength{\headsep}{10pt}
%\addtolength{\footskip}{10pt}

% Só pra ajustar o layout
%\setlength{\marginparwidth}{90pt}
\usepackage{layout}


% Para definir espaçamento entre as linhas
\usepackage{setspace}

% Espaçamento do texto para o frame
\setlength{\fboxsep}{1em}

% Faz com que as margens tenham o mesmo tamanho horizontalmente
%\geometry{hcentering}

%% Elementos Gráficos
% Para incluir figuras (pacote extendido)
\usepackage[]{graphicx}

% Suporte a cores
\usepackage{color}

% Criar figura dividida em subfiguras
% \usepackage{subfig}
% \captionsetup[subfigure]{style=default, margin=0pt, parskip=0pt, hangindent=0pt, indention=0pt, singlelinecheck=true, labelformat=parens, labelsep=space}

% Caso queira guardar as figuras em uma pasta separada
% (descomente e) defina o caminho para o diretório:
%\graphicspath{{./figuras/}}

% Customizar as legendas de figuras e tabelas
%\usepackage{caption}

% Criar ambientes com 2 ou mais colunas
%\usepackage{multicol}

% Ative o comando abaixo se quiser colocar figuras de fundo (e.g., capa)
\usepackage{wallpaper}
% Exemplo para inserir a figura na capa está no arquivo pre.tex (linha 7)
% Ajuste da posição da figura no eixo Y
\addtolength{\wpYoffset}{-0.25 \paperheight}
% Ajuste da posição da figura no eixo X
%\addtolength{\wpXoffset}{36pt}

%% Tabelas
% Elementos extras para formatação de tabelas
%\usepackage{array}

% Tabelas com qualidade de publicação
%\usepackage{booktabs}

% Para criar tabelas maiores que uma página
%\usepackage{longtable}

% adicionar tabelas e figuras como landscape
%\usepackage{lscape}

%% Lista de Abreviações
% Cria lista de abreviações
\usepackage[notintoc,portuguese]{nomencl}
\makenomenclature

%% Notas de rodapé
% Lidar com notas de rodapé em diversas situações
%\usepackage{footnote}
%\usepackage{footmisc}

% Notas criadas nas tabelas ficam no fim das tabelas
%\makesavenoteenv{tabular}

%% Links dinâmicos
% Suporte para hipertexto, links para referências e figuras
\usepackage{hyperref}
% Configurações dos links e metatags do PDF a ser gerado
\hypersetup{colorlinks=true, linkcolor=blue, citecolor=red, filecolor=red, pagecolor=red, urlcolor=blue}
            % pdfauthor={Nome do Autor},
            % pdftitle={Título do Projeto},
            % pdfsubject={Assunto do Projeto},
            % pdfkeywords={palavra-chave, palavra-chave, palavra-chave},
            % pdfproducer={Latex},
            % pdfcreator={pdflatex}}

% Conta o número de páginas
%\usepackage{lastpage}

%% Referências bibliográficas e afins
% Formatar as citações no texto e a lista de referências
%\usepackage{natbib}

% Adicionar bibliografia, índice e conteúdo na Tabela de conteúdo
% Não inclui lista de tabelas e figuras no índice
%\usepackage[nottoc,notlof,notlot]{tocbibind}

%% Pontuação e unidades
% Posicionar inteligentemente a vírgula como separador decimal
\usepackage{icomma}

% Formatar as unidades com as distâncias corretas
\usepackage[tight]{units}

% Sem numero de capitulo nas secoes
\setcounter{secnumdepth}{0}
% Eliminar o zero antes de cada numero de seção
\renewcommand*\thesection{\arabic{section}}

%Para eliminar palavra "capitulo" antes de cada capítulo
%dica de http://www.latex-community.org/forum/viewtopic.php?f=4&t=8522
    \makeatletter
    \renewcommand{\@makechapterhead}[1]{%
    \vspace*{50 pt}%
    {\setlength{\parindent}{0pt} \raggedright \normalfont
    \bfseries\Huge\thechapter.\ #1
    \par\nobreak\vspace{40 pt}}}
    \makeatother

%% Cabeçalho e rodapé
% Controlar os cabeçalhos e rodapés
%\usepackage{fancyhdr}

% Usar os estilos do pacote fancyhdr
%\pagestyle{fancy}

%Eliminar numeros do nome da seção e do capítulo que vao no cabeçalho
\renewcommand{\sectionmark}[1]{\markright{#1}{}}
\renewcommand{\chaptermark}[1]{\markboth{#1}{}}

% pagina sem headers para inicio de capitulos
% \fancypagestyle{plain}{
%         \fancyhf{}      
%         \cfoot{}
%         fancyfoot[RO,LE]{\LARGE \bfseries \thepage}    
%         }



% pagina sem nada
%\fancypagestyle{empty}{\fancyhf{}}
        
% Limpar os campos do cabeçalho atual
%\fancyhead{}

% Número da página do lado esquerdo [R] nas páginas ímpares [O] e do lado direito [L] nas páginas pares [E]
%\fancyhead[RO,LE]{\thepage}

%Eliminar a palavra capítulo do início dos capítulos
%\renewcommand{\chaptername}{} % nao funfa

% Nome da seção do lado direito em páginas ímpares
%\fancyhead[LO]{\large Manual PG-Ecologia USP - \the\year}
% Titulo do lado esquerdo em páginas pares
%\fancyhead[RE]{\large \leftmark}
% Páginas no rodapé
%\cfoot{}
%\fancyfoot[RO,LE]{\thepage}


% Para ter numeros de paginas mesmo na pagina inicial do capítulo


% linha de separação entre cabeçalho e conteúdo (coloque 0pt para omitir)
%\renewcommand{\headrulewidth}{0pt}
% linha de separação entre rodapé e conteúdo (coloque 0pt para omitir)
%\renewcommand{\footrulewidth}{0pt}
% Altura do cabeçalho
%\headheight 13.6pt

%% Inserir comentários no texto
% Marcar mudanças e fazer comentários
%\usepackage[margins]{trackchanges}
% Iniciais do autor
%\renewcommand{\initialsTwo}{bcv}
% Notas na margem interna
%\reversemarginpar

%% Comandos customizados

% % Espécie e abreviação
% \newcommand{\subde}{\emph{Clypeaster subdepressus}}
% \newcommand{\subsus}{\emph{C.~subdepressus}}

% % Título do projeto
% \newcommand{\titulo}{Título original do projeto}
% \newcommand{\nomedoaluno}{Nome Completo do Aluno}

%% Pacotes não implementados
% Para não sobrar espaços em branco estranhos
%\widowpenalty=1000
%\clubpenalty=1000
